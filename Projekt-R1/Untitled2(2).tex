\documentclass[11pt]{article}

    \usepackage[breakable]{tcolorbox}
    \usepackage{parskip} % Stop auto-indenting (to mimic markdown behaviour)
    
    \usepackage{iftex}
    \ifPDFTeX
    	\usepackage[T1]{fontenc}
    	\usepackage{mathpazo}
    \else
    	\usepackage{fontspec}
    \fi

    % Basic figure setup, for now with no caption control since it's done
    % automatically by Pandoc (which extracts ![](path) syntax from Markdown).
    \usepackage{graphicx}
    % Maintain compatibility with old templates. Remove in nbconvert 6.0
    \let\Oldincludegraphics\includegraphics
    % Ensure that by default, figures have no caption (until we provide a
    % proper Figure object with a Caption API and a way to capture that
    % in the conversion process - todo).
    \usepackage{caption}
    \DeclareCaptionFormat{nocaption}{}
    \captionsetup{format=nocaption,aboveskip=0pt,belowskip=0pt}

    \usepackage[Export]{adjustbox} % Used to constrain images to a maximum size
    \adjustboxset{max size={0.9\linewidth}{0.9\paperheight}}
    \usepackage{float}
    \floatplacement{figure}{H} % forces figures to be placed at the correct location
    \usepackage{xcolor} % Allow colors to be defined
    \usepackage{enumerate} % Needed for markdown enumerations to work
    \usepackage{geometry} % Used to adjust the document margins
    \usepackage{amsmath} % Equations
    \usepackage{amssymb} % Equations
    \usepackage{textcomp} % defines textquotesingle
    % Hack from http://tex.stackexchange.com/a/47451/13684:
    \AtBeginDocument{%
        \def\PYZsq{\textquotesingle}% Upright quotes in Pygmentized code
    }
    \usepackage{upquote} % Upright quotes for verbatim code
    \usepackage{eurosym} % defines \euro
    \usepackage[mathletters]{ucs} % Extended unicode (utf-8) support
    \usepackage{fancyvrb} % verbatim replacement that allows latex
    \usepackage{grffile} % extends the file name processing of package graphics 
                         % to support a larger range
    \makeatletter % fix for grffile with XeLaTeX
    \def\Gread@@xetex#1{%
      \IfFileExists{"\Gin@base".bb}%
      {\Gread@eps{\Gin@base.bb}}%
      {\Gread@@xetex@aux#1}%
    }
    \makeatother

    % The hyperref package gives us a pdf with properly built
    % internal navigation ('pdf bookmarks' for the table of contents,
    % internal cross-reference links, web links for URLs, etc.)
    \usepackage{hyperref}
    % The default LaTeX title has an obnoxious amount of whitespace. By default,
    % titling removes some of it. It also provides customization options.
    \usepackage{titling}
    \usepackage{longtable} % longtable support required by pandoc >1.10
    \usepackage{booktabs}  % table support for pandoc > 1.12.2
    \usepackage[inline]{enumitem} % IRkernel/repr support (it uses the enumerate* environment)
    \usepackage[normalem]{ulem} % ulem is needed to support strikethroughs (\sout)
                                % normalem makes italics be italics, not underlines
    \usepackage{mathrsfs}
    

    
    % Colors for the hyperref package
    \definecolor{urlcolor}{rgb}{0,.145,.698}
    \definecolor{linkcolor}{rgb}{.71,0.21,0.01}
    \definecolor{citecolor}{rgb}{.12,.54,.11}

    % ANSI colors
    \definecolor{ansi-black}{HTML}{3E424D}
    \definecolor{ansi-black-intense}{HTML}{282C36}
    \definecolor{ansi-red}{HTML}{E75C58}
    \definecolor{ansi-red-intense}{HTML}{B22B31}
    \definecolor{ansi-green}{HTML}{00A250}
    \definecolor{ansi-green-intense}{HTML}{007427}
    \definecolor{ansi-yellow}{HTML}{DDB62B}
    \definecolor{ansi-yellow-intense}{HTML}{B27D12}
    \definecolor{ansi-blue}{HTML}{208FFB}
    \definecolor{ansi-blue-intense}{HTML}{0065CA}
    \definecolor{ansi-magenta}{HTML}{D160C4}
    \definecolor{ansi-magenta-intense}{HTML}{A03196}
    \definecolor{ansi-cyan}{HTML}{60C6C8}
    \definecolor{ansi-cyan-intense}{HTML}{258F8F}
    \definecolor{ansi-white}{HTML}{C5C1B4}
    \definecolor{ansi-white-intense}{HTML}{A1A6B2}
    \definecolor{ansi-default-inverse-fg}{HTML}{FFFFFF}
    \definecolor{ansi-default-inverse-bg}{HTML}{000000}

    % commands and environments needed by pandoc snippets
    % extracted from the output of `pandoc -s`
    \providecommand{\tightlist}{%
      \setlength{\itemsep}{0pt}\setlength{\parskip}{0pt}}
    \DefineVerbatimEnvironment{Highlighting}{Verbatim}{commandchars=\\\{\}}
    % Add ',fontsize=\small' for more characters per line
    \newenvironment{Shaded}{}{}
    \newcommand{\KeywordTok}[1]{\textcolor[rgb]{0.00,0.44,0.13}{\textbf{{#1}}}}
    \newcommand{\DataTypeTok}[1]{\textcolor[rgb]{0.56,0.13,0.00}{{#1}}}
    \newcommand{\DecValTok}[1]{\textcolor[rgb]{0.25,0.63,0.44}{{#1}}}
    \newcommand{\BaseNTok}[1]{\textcolor[rgb]{0.25,0.63,0.44}{{#1}}}
    \newcommand{\FloatTok}[1]{\textcolor[rgb]{0.25,0.63,0.44}{{#1}}}
    \newcommand{\CharTok}[1]{\textcolor[rgb]{0.25,0.44,0.63}{{#1}}}
    \newcommand{\StringTok}[1]{\textcolor[rgb]{0.25,0.44,0.63}{{#1}}}
    \newcommand{\CommentTok}[1]{\textcolor[rgb]{0.38,0.63,0.69}{\textit{{#1}}}}
    \newcommand{\OtherTok}[1]{\textcolor[rgb]{0.00,0.44,0.13}{{#1}}}
    \newcommand{\AlertTok}[1]{\textcolor[rgb]{1.00,0.00,0.00}{\textbf{{#1}}}}
    \newcommand{\FunctionTok}[1]{\textcolor[rgb]{0.02,0.16,0.49}{{#1}}}
    \newcommand{\RegionMarkerTok}[1]{{#1}}
    \newcommand{\ErrorTok}[1]{\textcolor[rgb]{1.00,0.00,0.00}{\textbf{{#1}}}}
    \newcommand{\NormalTok}[1]{{#1}}
    
    % Additional commands for more recent versions of Pandoc
    \newcommand{\ConstantTok}[1]{\textcolor[rgb]{0.53,0.00,0.00}{{#1}}}
    \newcommand{\SpecialCharTok}[1]{\textcolor[rgb]{0.25,0.44,0.63}{{#1}}}
    \newcommand{\VerbatimStringTok}[1]{\textcolor[rgb]{0.25,0.44,0.63}{{#1}}}
    \newcommand{\SpecialStringTok}[1]{\textcolor[rgb]{0.73,0.40,0.53}{{#1}}}
    \newcommand{\ImportTok}[1]{{#1}}
    \newcommand{\DocumentationTok}[1]{\textcolor[rgb]{0.73,0.13,0.13}{\textit{{#1}}}}
    \newcommand{\AnnotationTok}[1]{\textcolor[rgb]{0.38,0.63,0.69}{\textbf{\textit{{#1}}}}}
    \newcommand{\CommentVarTok}[1]{\textcolor[rgb]{0.38,0.63,0.69}{\textbf{\textit{{#1}}}}}
    \newcommand{\VariableTok}[1]{\textcolor[rgb]{0.10,0.09,0.49}{{#1}}}
    \newcommand{\ControlFlowTok}[1]{\textcolor[rgb]{0.00,0.44,0.13}{\textbf{{#1}}}}
    \newcommand{\OperatorTok}[1]{\textcolor[rgb]{0.40,0.40,0.40}{{#1}}}
    \newcommand{\BuiltInTok}[1]{{#1}}
    \newcommand{\ExtensionTok}[1]{{#1}}
    \newcommand{\PreprocessorTok}[1]{\textcolor[rgb]{0.74,0.48,0.00}{{#1}}}
    \newcommand{\AttributeTok}[1]{\textcolor[rgb]{0.49,0.56,0.16}{{#1}}}
    \newcommand{\InformationTok}[1]{\textcolor[rgb]{0.38,0.63,0.69}{\textbf{\textit{{#1}}}}}
    \newcommand{\WarningTok}[1]{\textcolor[rgb]{0.38,0.63,0.69}{\textbf{\textit{{#1}}}}}
    
    
    % Define a nice break command that doesn't care if a line doesn't already
    % exist.
    \def\br{\hspace*{\fill} \\* }
    % Math Jax compatibility definitions
    \def\gt{>}
    \def\lt{<}
    \let\Oldtex\TeX
    \let\Oldlatex\LaTeX
    \renewcommand{\TeX}{\textrm{\Oldtex}}
    \renewcommand{\LaTeX}{\textrm{\Oldlatex}}
    % Document parameters
    % Document title
    \title{Zmiany}
    
    
    
    
    
% Pygments definitions
\makeatletter
\def\PY@reset{\let\PY@it=\relax \let\PY@bf=\relax%
    \let\PY@ul=\relax \let\PY@tc=\relax%
    \let\PY@bc=\relax \let\PY@ff=\relax}
\def\PY@tok#1{\csname PY@tok@#1\endcsname}
\def\PY@toks#1+{\ifx\relax#1\empty\else%
    \PY@tok{#1}\expandafter\PY@toks\fi}
\def\PY@do#1{\PY@bc{\PY@tc{\PY@ul{%
    \PY@it{\PY@bf{\PY@ff{#1}}}}}}}
\def\PY#1#2{\PY@reset\PY@toks#1+\relax+\PY@do{#2}}

\expandafter\def\csname PY@tok@w\endcsname{\def\PY@tc##1{\textcolor[rgb]{0.73,0.73,0.73}{##1}}}
\expandafter\def\csname PY@tok@c\endcsname{\let\PY@it=\textit\def\PY@tc##1{\textcolor[rgb]{0.25,0.50,0.50}{##1}}}
\expandafter\def\csname PY@tok@cp\endcsname{\def\PY@tc##1{\textcolor[rgb]{0.74,0.48,0.00}{##1}}}
\expandafter\def\csname PY@tok@k\endcsname{\let\PY@bf=\textbf\def\PY@tc##1{\textcolor[rgb]{0.00,0.50,0.00}{##1}}}
\expandafter\def\csname PY@tok@kp\endcsname{\def\PY@tc##1{\textcolor[rgb]{0.00,0.50,0.00}{##1}}}
\expandafter\def\csname PY@tok@kt\endcsname{\def\PY@tc##1{\textcolor[rgb]{0.69,0.00,0.25}{##1}}}
\expandafter\def\csname PY@tok@o\endcsname{\def\PY@tc##1{\textcolor[rgb]{0.40,0.40,0.40}{##1}}}
\expandafter\def\csname PY@tok@ow\endcsname{\let\PY@bf=\textbf\def\PY@tc##1{\textcolor[rgb]{0.67,0.13,1.00}{##1}}}
\expandafter\def\csname PY@tok@nb\endcsname{\def\PY@tc##1{\textcolor[rgb]{0.00,0.50,0.00}{##1}}}
\expandafter\def\csname PY@tok@nf\endcsname{\def\PY@tc##1{\textcolor[rgb]{0.00,0.00,1.00}{##1}}}
\expandafter\def\csname PY@tok@nc\endcsname{\let\PY@bf=\textbf\def\PY@tc##1{\textcolor[rgb]{0.00,0.00,1.00}{##1}}}
\expandafter\def\csname PY@tok@nn\endcsname{\let\PY@bf=\textbf\def\PY@tc##1{\textcolor[rgb]{0.00,0.00,1.00}{##1}}}
\expandafter\def\csname PY@tok@ne\endcsname{\let\PY@bf=\textbf\def\PY@tc##1{\textcolor[rgb]{0.82,0.25,0.23}{##1}}}
\expandafter\def\csname PY@tok@nv\endcsname{\def\PY@tc##1{\textcolor[rgb]{0.10,0.09,0.49}{##1}}}
\expandafter\def\csname PY@tok@no\endcsname{\def\PY@tc##1{\textcolor[rgb]{0.53,0.00,0.00}{##1}}}
\expandafter\def\csname PY@tok@nl\endcsname{\def\PY@tc##1{\textcolor[rgb]{0.63,0.63,0.00}{##1}}}
\expandafter\def\csname PY@tok@ni\endcsname{\let\PY@bf=\textbf\def\PY@tc##1{\textcolor[rgb]{0.60,0.60,0.60}{##1}}}
\expandafter\def\csname PY@tok@na\endcsname{\def\PY@tc##1{\textcolor[rgb]{0.49,0.56,0.16}{##1}}}
\expandafter\def\csname PY@tok@nt\endcsname{\let\PY@bf=\textbf\def\PY@tc##1{\textcolor[rgb]{0.00,0.50,0.00}{##1}}}
\expandafter\def\csname PY@tok@nd\endcsname{\def\PY@tc##1{\textcolor[rgb]{0.67,0.13,1.00}{##1}}}
\expandafter\def\csname PY@tok@s\endcsname{\def\PY@tc##1{\textcolor[rgb]{0.73,0.13,0.13}{##1}}}
\expandafter\def\csname PY@tok@sd\endcsname{\let\PY@it=\textit\def\PY@tc##1{\textcolor[rgb]{0.73,0.13,0.13}{##1}}}
\expandafter\def\csname PY@tok@si\endcsname{\let\PY@bf=\textbf\def\PY@tc##1{\textcolor[rgb]{0.73,0.40,0.53}{##1}}}
\expandafter\def\csname PY@tok@se\endcsname{\let\PY@bf=\textbf\def\PY@tc##1{\textcolor[rgb]{0.73,0.40,0.13}{##1}}}
\expandafter\def\csname PY@tok@sr\endcsname{\def\PY@tc##1{\textcolor[rgb]{0.73,0.40,0.53}{##1}}}
\expandafter\def\csname PY@tok@ss\endcsname{\def\PY@tc##1{\textcolor[rgb]{0.10,0.09,0.49}{##1}}}
\expandafter\def\csname PY@tok@sx\endcsname{\def\PY@tc##1{\textcolor[rgb]{0.00,0.50,0.00}{##1}}}
\expandafter\def\csname PY@tok@m\endcsname{\def\PY@tc##1{\textcolor[rgb]{0.40,0.40,0.40}{##1}}}
\expandafter\def\csname PY@tok@gh\endcsname{\let\PY@bf=\textbf\def\PY@tc##1{\textcolor[rgb]{0.00,0.00,0.50}{##1}}}
\expandafter\def\csname PY@tok@gu\endcsname{\let\PY@bf=\textbf\def\PY@tc##1{\textcolor[rgb]{0.50,0.00,0.50}{##1}}}
\expandafter\def\csname PY@tok@gd\endcsname{\def\PY@tc##1{\textcolor[rgb]{0.63,0.00,0.00}{##1}}}
\expandafter\def\csname PY@tok@gi\endcsname{\def\PY@tc##1{\textcolor[rgb]{0.00,0.63,0.00}{##1}}}
\expandafter\def\csname PY@tok@gr\endcsname{\def\PY@tc##1{\textcolor[rgb]{1.00,0.00,0.00}{##1}}}
\expandafter\def\csname PY@tok@ge\endcsname{\let\PY@it=\textit}
\expandafter\def\csname PY@tok@gs\endcsname{\let\PY@bf=\textbf}
\expandafter\def\csname PY@tok@gp\endcsname{\let\PY@bf=\textbf\def\PY@tc##1{\textcolor[rgb]{0.00,0.00,0.50}{##1}}}
\expandafter\def\csname PY@tok@go\endcsname{\def\PY@tc##1{\textcolor[rgb]{0.53,0.53,0.53}{##1}}}
\expandafter\def\csname PY@tok@gt\endcsname{\def\PY@tc##1{\textcolor[rgb]{0.00,0.27,0.87}{##1}}}
\expandafter\def\csname PY@tok@err\endcsname{\def\PY@bc##1{\setlength{\fboxsep}{0pt}\fcolorbox[rgb]{1.00,0.00,0.00}{1,1,1}{\strut ##1}}}
\expandafter\def\csname PY@tok@kc\endcsname{\let\PY@bf=\textbf\def\PY@tc##1{\textcolor[rgb]{0.00,0.50,0.00}{##1}}}
\expandafter\def\csname PY@tok@kd\endcsname{\let\PY@bf=\textbf\def\PY@tc##1{\textcolor[rgb]{0.00,0.50,0.00}{##1}}}
\expandafter\def\csname PY@tok@kn\endcsname{\let\PY@bf=\textbf\def\PY@tc##1{\textcolor[rgb]{0.00,0.50,0.00}{##1}}}
\expandafter\def\csname PY@tok@kr\endcsname{\let\PY@bf=\textbf\def\PY@tc##1{\textcolor[rgb]{0.00,0.50,0.00}{##1}}}
\expandafter\def\csname PY@tok@bp\endcsname{\def\PY@tc##1{\textcolor[rgb]{0.00,0.50,0.00}{##1}}}
\expandafter\def\csname PY@tok@fm\endcsname{\def\PY@tc##1{\textcolor[rgb]{0.00,0.00,1.00}{##1}}}
\expandafter\def\csname PY@tok@vc\endcsname{\def\PY@tc##1{\textcolor[rgb]{0.10,0.09,0.49}{##1}}}
\expandafter\def\csname PY@tok@vg\endcsname{\def\PY@tc##1{\textcolor[rgb]{0.10,0.09,0.49}{##1}}}
\expandafter\def\csname PY@tok@vi\endcsname{\def\PY@tc##1{\textcolor[rgb]{0.10,0.09,0.49}{##1}}}
\expandafter\def\csname PY@tok@vm\endcsname{\def\PY@tc##1{\textcolor[rgb]{0.10,0.09,0.49}{##1}}}
\expandafter\def\csname PY@tok@sa\endcsname{\def\PY@tc##1{\textcolor[rgb]{0.73,0.13,0.13}{##1}}}
\expandafter\def\csname PY@tok@sb\endcsname{\def\PY@tc##1{\textcolor[rgb]{0.73,0.13,0.13}{##1}}}
\expandafter\def\csname PY@tok@sc\endcsname{\def\PY@tc##1{\textcolor[rgb]{0.73,0.13,0.13}{##1}}}
\expandafter\def\csname PY@tok@dl\endcsname{\def\PY@tc##1{\textcolor[rgb]{0.73,0.13,0.13}{##1}}}
\expandafter\def\csname PY@tok@s2\endcsname{\def\PY@tc##1{\textcolor[rgb]{0.73,0.13,0.13}{##1}}}
\expandafter\def\csname PY@tok@sh\endcsname{\def\PY@tc##1{\textcolor[rgb]{0.73,0.13,0.13}{##1}}}
\expandafter\def\csname PY@tok@s1\endcsname{\def\PY@tc##1{\textcolor[rgb]{0.73,0.13,0.13}{##1}}}
\expandafter\def\csname PY@tok@mb\endcsname{\def\PY@tc##1{\textcolor[rgb]{0.40,0.40,0.40}{##1}}}
\expandafter\def\csname PY@tok@mf\endcsname{\def\PY@tc##1{\textcolor[rgb]{0.40,0.40,0.40}{##1}}}
\expandafter\def\csname PY@tok@mh\endcsname{\def\PY@tc##1{\textcolor[rgb]{0.40,0.40,0.40}{##1}}}
\expandafter\def\csname PY@tok@mi\endcsname{\def\PY@tc##1{\textcolor[rgb]{0.40,0.40,0.40}{##1}}}
\expandafter\def\csname PY@tok@il\endcsname{\def\PY@tc##1{\textcolor[rgb]{0.40,0.40,0.40}{##1}}}
\expandafter\def\csname PY@tok@mo\endcsname{\def\PY@tc##1{\textcolor[rgb]{0.40,0.40,0.40}{##1}}}
\expandafter\def\csname PY@tok@ch\endcsname{\let\PY@it=\textit\def\PY@tc##1{\textcolor[rgb]{0.25,0.50,0.50}{##1}}}
\expandafter\def\csname PY@tok@cm\endcsname{\let\PY@it=\textit\def\PY@tc##1{\textcolor[rgb]{0.25,0.50,0.50}{##1}}}
\expandafter\def\csname PY@tok@cpf\endcsname{\let\PY@it=\textit\def\PY@tc##1{\textcolor[rgb]{0.25,0.50,0.50}{##1}}}
\expandafter\def\csname PY@tok@c1\endcsname{\let\PY@it=\textit\def\PY@tc##1{\textcolor[rgb]{0.25,0.50,0.50}{##1}}}
\expandafter\def\csname PY@tok@cs\endcsname{\let\PY@it=\textit\def\PY@tc##1{\textcolor[rgb]{0.25,0.50,0.50}{##1}}}

\def\PYZbs{\char`\\}
\def\PYZus{\char`\_}
\def\PYZob{\char`\{}
\def\PYZcb{\char`\}}
\def\PYZca{\char`\^}
\def\PYZam{\char`\&}
\def\PYZlt{\char`\<}
\def\PYZgt{\char`\>}
\def\PYZsh{\char`\#}
\def\PYZpc{\char`\%}
\def\PYZdl{\char`\$}
\def\PYZhy{\char`\-}
\def\PYZsq{\char`\'}
\def\PYZdq{\char`\"}
\def\PYZti{\char`\~}
% for compatibility with earlier versions
\def\PYZat{@}
\def\PYZlb{[}
\def\PYZrb{]}
\makeatother


    % For linebreaks inside Verbatim environment from package fancyvrb. 
    \makeatletter
        \newbox\Wrappedcontinuationbox 
        \newbox\Wrappedvisiblespacebox 
        \newcommand*\Wrappedvisiblespace {\textcolor{red}{\textvisiblespace}} 
        \newcommand*\Wrappedcontinuationsymbol {\textcolor{red}{\llap{\tiny$\m@th\hookrightarrow$}}} 
        \newcommand*\Wrappedcontinuationindent {3ex } 
        \newcommand*\Wrappedafterbreak {\kern\Wrappedcontinuationindent\copy\Wrappedcontinuationbox} 
        % Take advantage of the already applied Pygments mark-up to insert 
        % potential linebreaks for TeX processing. 
        %        {, <, #, %, $, ' and ": go to next line. 
        %        _, }, ^, &, >, - and ~: stay at end of broken line. 
        % Use of \textquotesingle for straight quote. 
        \newcommand*\Wrappedbreaksatspecials {% 
            \def\PYGZus{\discretionary{\char`\_}{\Wrappedafterbreak}{\char`\_}}% 
            \def\PYGZob{\discretionary{}{\Wrappedafterbreak\char`\{}{\char`\{}}% 
            \def\PYGZcb{\discretionary{\char`\}}{\Wrappedafterbreak}{\char`\}}}% 
            \def\PYGZca{\discretionary{\char`\^}{\Wrappedafterbreak}{\char`\^}}% 
            \def\PYGZam{\discretionary{\char`\&}{\Wrappedafterbreak}{\char`\&}}% 
            \def\PYGZlt{\discretionary{}{\Wrappedafterbreak\char`\<}{\char`\<}}% 
            \def\PYGZgt{\discretionary{\char`\>}{\Wrappedafterbreak}{\char`\>}}% 
            \def\PYGZsh{\discretionary{}{\Wrappedafterbreak\char`\#}{\char`\#}}% 
            \def\PYGZpc{\discretionary{}{\Wrappedafterbreak\char`\%}{\char`\%}}% 
            \def\PYGZdl{\discretionary{}{\Wrappedafterbreak\char`\$}{\char`\$}}% 
            \def\PYGZhy{\discretionary{\char`\-}{\Wrappedafterbreak}{\char`\-}}% 
            \def\PYGZsq{\discretionary{}{\Wrappedafterbreak\textquotesingle}{\textquotesingle}}% 
            \def\PYGZdq{\discretionary{}{\Wrappedafterbreak\char`\"}{\char`\"}}% 
            \def\PYGZti{\discretionary{\char`\~}{\Wrappedafterbreak}{\char`\~}}% 
        } 
        % Some characters . , ; ? ! / are not pygmentized. 
        % This macro makes them "active" and they will insert potential linebreaks 
        \newcommand*\Wrappedbreaksatpunct {% 
            \lccode`\~`\.\lowercase{\def~}{\discretionary{\hbox{\char`\.}}{\Wrappedafterbreak}{\hbox{\char`\.}}}% 
            \lccode`\~`\,\lowercase{\def~}{\discretionary{\hbox{\char`\,}}{\Wrappedafterbreak}{\hbox{\char`\,}}}% 
            \lccode`\~`\;\lowercase{\def~}{\discretionary{\hbox{\char`\;}}{\Wrappedafterbreak}{\hbox{\char`\;}}}% 
            \lccode`\~`\:\lowercase{\def~}{\discretionary{\hbox{\char`\:}}{\Wrappedafterbreak}{\hbox{\char`\:}}}% 
            \lccode`\~`\?\lowercase{\def~}{\discretionary{\hbox{\char`\?}}{\Wrappedafterbreak}{\hbox{\char`\?}}}% 
            \lccode`\~`\!\lowercase{\def~}{\discretionary{\hbox{\char`\!}}{\Wrappedafterbreak}{\hbox{\char`\!}}}% 
            \lccode`\~`\/\lowercase{\def~}{\discretionary{\hbox{\char`\/}}{\Wrappedafterbreak}{\hbox{\char`\/}}}% 
            \catcode`\.\active
            \catcode`\,\active 
            \catcode`\;\active
            \catcode`\:\active
            \catcode`\?\active
            \catcode`\!\active
            \catcode`\/\active 
            \lccode`\~`\~ 	
        }
    \makeatother

    \let\OriginalVerbatim=\Verbatim
    \makeatletter
    \renewcommand{\Verbatim}[1][1]{%
        %\parskip\z@skip
        \sbox\Wrappedcontinuationbox {\Wrappedcontinuationsymbol}%
        \sbox\Wrappedvisiblespacebox {\FV@SetupFont\Wrappedvisiblespace}%
        \def\FancyVerbFormatLine ##1{\hsize\linewidth
            \vtop{\raggedright\hyphenpenalty\z@\exhyphenpenalty\z@
                \doublehyphendemerits\z@\finalhyphendemerits\z@
                \strut ##1\strut}%
        }%
        % If the linebreak is at a space, the latter will be displayed as visible
        % space at end of first line, and a continuation symbol starts next line.
        % Stretch/shrink are however usually zero for typewriter font.
        \def\FV@Space {%
            \nobreak\hskip\z@ plus\fontdimen3\font minus\fontdimen4\font
            \discretionary{\copy\Wrappedvisiblespacebox}{\Wrappedafterbreak}
            {\kern\fontdimen2\font}%
        }%
        
        % Allow breaks at special characters using \PYG... macros.
        \Wrappedbreaksatspecials
        % Breaks at punctuation characters . , ; ? ! and / need catcode=\active 	
        \OriginalVerbatim[#1,codes*=\Wrappedbreaksatpunct]%
    }
    \makeatother

    % Exact colors from NB
    \definecolor{incolor}{HTML}{303F9F}
    \definecolor{outcolor}{HTML}{D84315}
    \definecolor{cellborder}{HTML}{CFCFCF}
    \definecolor{cellbackground}{HTML}{F7F7F7}
    
    % prompt
    \makeatletter
    \newcommand{\boxspacing}{\kern\kvtcb@left@rule\kern\kvtcb@boxsep}
    \makeatother
    \newcommand{\prompt}[4]{
        \ttfamily\llap{{\color{#2}[#3]:\hspace{3pt}#4}}\vspace{-\baselineskip}
    }
    

    
    % Prevent overflowing lines due to hard-to-break entities
    \sloppy 
    % Setup hyperref package
    \hypersetup{
      breaklinks=true,  % so long urls are correctly broken across lines
      colorlinks=true,
      urlcolor=urlcolor,
      linkcolor=linkcolor,
      citecolor=citecolor,
      }
    % Slightly bigger margins than the latex defaults
    
    \geometry{verbose,tmargin=1in,bmargin=1in,lmargin=1in,rmargin=1in}
    
    

\begin{document}
    
    \maketitle
    
    

    
    \begin{tcolorbox}[breakable, size=fbox, boxrule=1pt, pad at break*=1mm,colback=cellbackground, colframe=cellborder]
\prompt{In}{incolor}{1}{\boxspacing}
\begin{Verbatim}[commandchars=\\\{\}]
\PY{n+nf}{setwd}\PY{p}{(}\PY{l+s}{\PYZdq{}}\PY{l+s}{/home/makbet/ProjektR\PYZdq{}}\PY{p}{)}
\end{Verbatim}
\end{tcolorbox}

    \hypertarget{pierwszym-krokiem-bux119dzie-ux15bciux105gniux119cie-niezbux119dnych-bibliotek-ktuxf3re-pomogux105-stworzyux107-odpowiednie-pliki-oraz-wykresy}{%
\subparagraph{Pierwszym krokiem będzie ściągnięcie niezbędnych
bibliotek, które pomogą stworzyć odpowiednie pliki oraz
wykresy}\label{pierwszym-krokiem-bux119dzie-ux15bciux105gniux119cie-niezbux119dnych-bibliotek-ktuxf3re-pomogux105-stworzyux107-odpowiednie-pliki-oraz-wykresy}}

    \begin{tcolorbox}[breakable, size=fbox, boxrule=1pt, pad at break*=1mm,colback=cellbackground, colframe=cellborder]
\prompt{In}{incolor}{10}{\boxspacing}
\begin{Verbatim}[commandchars=\\\{\}]
\PY{n+nf}{library}\PY{p}{(}\PY{n}{xlsx}\PY{p}{)}
  \PY{n+nf}{library}\PY{p}{(}\PY{n}{ggplot2}\PY{p}{)}
  \PY{n+nf}{library}\PY{p}{(}\PY{n}{dplyr}\PY{p}{)}
  \PY{n+nf}{library}\PY{p}{(}\PY{n}{hrbrthemes}\PY{p}{)}
  \PY{n+nf}{library}\PY{p}{(}\PY{n}{data.table}\PY{p}{)}
\end{Verbatim}
\end{tcolorbox}

    \begin{Verbatim}[commandchars=\\\{\}]
Warning message in install.packages("rJava"):
“installation of package ‘rJava’ had non-zero exit status”Updating HTML index of
packages in '.Library'
Making 'packages.html' {\ldots} done
    \end{Verbatim}

    \begin{Verbatim}[commandchars=\\\{\}]

        Error in library(xlsx): there is no package called ‘xlsx’
    Traceback:


        1. library(xlsx)

    \end{Verbatim}

    W tym etapie ładuję plik do odczytu i przekształcam go w data frame, by
ułatwić sobie pracę z danymi

    \begin{tcolorbox}[breakable, size=fbox, boxrule=1pt, pad at break*=1mm,colback=cellbackground, colframe=cellborder]
\prompt{In}{incolor}{ }{\boxspacing}
\begin{Verbatim}[commandchars=\\\{\}]
\PY{n}{dane}\PY{o}{\PYZlt{}\PYZhy{}}\PY{n+nf}{read.xlsx}\PY{p}{(}\PY{l+s}{\PYZdq{}}\PY{l+s}{ceny.xlsx\PYZdq{}}\PY{p}{,} \PY{n}{sheetIndex}\PY{o}{=}\PY{l+m}{2}\PY{p}{,}\PY{n}{stringsAsFactors}\PY{o}{=}\PY{k+kc}{FALSE}\PY{p}{)}
  \PY{n}{dane}\PY{o}{\PYZlt{}\PYZhy{}}\PY{n+nf}{data.frame}\PY{p}{(}\PY{n}{dane}\PY{p}{)}
  \PY{n}{dane}\PY{o}{\PYZdl{}}\PY{n}{Wartosc}\PY{o}{\PYZlt{}\PYZhy{}}\PY{n+nf}{as.numeric}\PY{p}{(}\PY{n}{dane}\PY{o}{\PYZdl{}}\PY{n}{Wartosc}\PY{p}{)}
\end{Verbatim}
\end{tcolorbox}

    Ten etap jest jednym z najważniejszych

zbieram średnią cenę wszystkich towarów i łącze je w średnią rocznę
zakupu. Dane zapisuję do pliku

    \begin{tcolorbox}[breakable, size=fbox, boxrule=1pt, pad at break*=1mm,colback=cellbackground, colframe=cellborder]
\prompt{In}{incolor}{ }{\boxspacing}
\begin{Verbatim}[commandchars=\\\{\}]
\PY{n}{wojewodztwa\PYZus{}ceny}\PY{o}{\PYZlt{}\PYZhy{}}\PY{n+nf}{tapply}\PY{p}{(}\PY{n}{dane}\PY{o}{\PYZdl{}}\PY{n}{Wartosc}\PY{p}{,}\PY{n+nf}{list}\PY{p}{(}\PY{n}{dane}\PY{o}{\PYZdl{}}\PY{n}{Rok}\PY{p}{,}\PY{n}{dane}\PY{o}{\PYZdl{}}\PY{n}{Nazwa}\PY{p}{)}\PY{p}{,}\PY{n}{FUN}\PY{o}{=}\PY{n}{mean}\PY{p}{,}\PY{n}{na.rm}\PY{o}{=}\PY{k+kc}{TRUE}\PY{p}{)}
  \PY{n}{wojewodztwa\PYZus{}ceny}\PY{o}{\PYZlt{}\PYZhy{}}\PY{n+nf}{data.frame}\PY{p}{(}\PY{n}{wojewodztwa\PYZus{}ceny}\PY{p}{)}
  \PY{n+nf}{print}\PY{p}{(}\PY{n}{wojewodztwa\PYZus{}ceny}\PY{p}{)}
  \PY{n+nf}{write.xlsx}\PY{p}{(}\PY{p}{(}\PY{n+nf}{format}\PY{p}{(}\PY{n}{wojewodztwa\PYZus{}ceny}\PY{p}{,} \PY{n}{digits}\PY{o}{=}\PY{l+m}{3}\PY{p}{)}\PY{p}{)}\PY{p}{,}\PY{l+s}{\PYZdq{}}\PY{l+s}{Dane\PYZus{}usrednione.xlsx\PYZdq{}}\PY{p}{)}
\end{Verbatim}
\end{tcolorbox}

    Teraz policzymy srednia cene krajowa w ciagu kazdego roku

    \begin{tcolorbox}[breakable, size=fbox, boxrule=1pt, pad at break*=1mm,colback=cellbackground, colframe=cellborder]
\prompt{In}{incolor}{ }{\boxspacing}
\begin{Verbatim}[commandchars=\\\{\}]
\PY{n}{TowarRok}\PY{o}{\PYZlt{}\PYZhy{}}\PY{n+nf}{tapply}\PY{p}{(}\PY{n}{dane}\PY{o}{\PYZdl{}}\PY{n}{Wartosc}\PY{p}{,}\PY{n+nf}{list}\PY{p}{(}\PY{n}{dane}\PY{o}{\PYZdl{}}\PY{n}{Rok}\PY{p}{,}\PY{n}{dane}\PY{o}{\PYZdl{}}\PY{n}{Towar}\PY{p}{)}\PY{p}{,}\PY{n}{FUN}\PY{o}{=}\PY{n}{mean}\PY{p}{,}\PY{n}{na.rm}\PY{o}{=}\PY{k+kc}{TRUE}\PY{p}{)}
\PY{n}{TowarRok}\PY{o}{\PYZlt{}\PYZhy{}}\PY{n+nf}{data.frame}\PY{p}{(}\PY{n}{TowarRok}\PY{p}{,}\PY{n}{stringsAsFactors} \PY{o}{=} \PY{k+kc}{FALSE}\PY{p}{)}
\end{Verbatim}
\end{tcolorbox}

    \hypertarget{zbieram-ux15bredniux105-cenux119-z-kaux17cdego-roku-do-listy}{%
\subparagraph{Zbieram średnią cenę z każdego roku do
listy}\label{zbieram-ux15bredniux105-cenux119-z-kaux17cdego-roku-do-listy}}

    \begin{tcolorbox}[breakable, size=fbox, boxrule=1pt, pad at break*=1mm,colback=cellbackground, colframe=cellborder]
\prompt{In}{incolor}{ }{\boxspacing}
\begin{Verbatim}[commandchars=\\\{\}]
\PY{n}{lista\PYZus{}srednich\PYZus{}cen}\PY{o}{\PYZlt{}\PYZhy{}}\PY{n+nf}{c}\PY{p}{(}
  
  \PY{n}{Srednia2006}\PY{o}{\PYZlt{}\PYZhy{}}\PY{n+nf}{mean}\PY{p}{(}\PY{n+nf}{as.numeric}\PY{p}{(}\PY{n}{TowarRok}\PY{n+nf}{[c}\PY{p}{(}\PY{l+m}{1}\PY{p}{)}\PY{p}{,}\PY{n+nf}{c}\PY{p}{(}\PY{l+m}{1}\PY{o}{:}\PY{l+m}{10}\PY{p}{)}\PY{n}{]}\PY{p}{)}\PY{p}{,}\PY{n}{na.rm}\PY{o}{=}\PY{k+kc}{TRUE}\PY{p}{)}\PY{p}{,}

  \PY{n}{Srednia2007}\PY{o}{\PYZlt{}\PYZhy{}}\PY{n+nf}{mean}\PY{p}{(}\PY{n+nf}{as.numeric}\PY{p}{(}\PY{n}{TowarRok}\PY{n+nf}{[c}\PY{p}{(}\PY{l+m}{2}\PY{p}{)}\PY{p}{,}\PY{n+nf}{c}\PY{p}{(}\PY{l+m}{1}\PY{o}{:}\PY{l+m}{10}\PY{p}{)}\PY{n}{]}\PY{p}{)}\PY{p}{,}\PY{n}{na.rm}\PY{o}{=}\PY{k+kc}{TRUE}\PY{p}{)}\PY{p}{,}

  \PY{n}{Srednia2008}\PY{o}{\PYZlt{}\PYZhy{}}\PY{n+nf}{mean}\PY{p}{(}\PY{n+nf}{as.numeric}\PY{p}{(}\PY{n}{TowarRok}\PY{n+nf}{[c}\PY{p}{(}\PY{l+m}{3}\PY{p}{)}\PY{p}{,}\PY{n+nf}{c}\PY{p}{(}\PY{l+m}{1}\PY{o}{:}\PY{l+m}{10}\PY{p}{)}\PY{n}{]}\PY{p}{)}\PY{p}{,}\PY{n}{na.rm}\PY{o}{=}\PY{k+kc}{TRUE}\PY{p}{)}\PY{p}{,}

  \PY{n}{Srednia2009}\PY{o}{\PYZlt{}\PYZhy{}}\PY{n+nf}{mean}\PY{p}{(}\PY{n+nf}{as.numeric}\PY{p}{(}\PY{n}{TowarRok}\PY{n+nf}{[c}\PY{p}{(}\PY{l+m}{4}\PY{p}{)}\PY{p}{,}\PY{n+nf}{c}\PY{p}{(}\PY{l+m}{1}\PY{o}{:}\PY{l+m}{10}\PY{p}{)}\PY{n}{]}\PY{p}{)}\PY{p}{,}\PY{n}{na.rm}\PY{o}{=}\PY{k+kc}{TRUE}\PY{p}{)}\PY{p}{,}

  \PY{n}{Srednia2010}\PY{o}{\PYZlt{}\PYZhy{}}\PY{n+nf}{mean}\PY{p}{(}\PY{n+nf}{as.numeric}\PY{p}{(}\PY{n}{TowarRok}\PY{n+nf}{[c}\PY{p}{(}\PY{l+m}{5}\PY{p}{)}\PY{p}{,}\PY{n+nf}{c}\PY{p}{(}\PY{l+m}{1}\PY{o}{:}\PY{l+m}{10}\PY{p}{)}\PY{n}{]}\PY{p}{)}\PY{p}{,}\PY{n}{na.rm}\PY{o}{=}\PY{k+kc}{TRUE}\PY{p}{)}\PY{p}{,}

  \PY{n}{Srednia2011}\PY{o}{\PYZlt{}\PYZhy{}}\PY{n+nf}{mean}\PY{p}{(}\PY{n+nf}{as.numeric}\PY{p}{(}\PY{n}{TowarRok}\PY{n+nf}{[c}\PY{p}{(}\PY{l+m}{6}\PY{p}{)}\PY{p}{,}\PY{n+nf}{c}\PY{p}{(}\PY{l+m}{1}\PY{o}{:}\PY{l+m}{10}\PY{p}{)}\PY{n}{]}\PY{p}{)}\PY{p}{,}\PY{n}{na.rm}\PY{o}{=}\PY{k+kc}{TRUE}\PY{p}{)}\PY{p}{,}

  \PY{n}{Srednia2012}\PY{o}{\PYZlt{}\PYZhy{}}\PY{n+nf}{mean}\PY{p}{(}\PY{n+nf}{as.numeric}\PY{p}{(}\PY{n}{TowarRok}\PY{n+nf}{[c}\PY{p}{(}\PY{l+m}{7}\PY{p}{)}\PY{p}{,}\PY{n+nf}{c}\PY{p}{(}\PY{l+m}{1}\PY{o}{:}\PY{l+m}{10}\PY{p}{)}\PY{n}{]}\PY{p}{)}\PY{p}{,}\PY{n}{na.rm}\PY{o}{=}\PY{k+kc}{TRUE}\PY{p}{)}\PY{p}{,}

  \PY{n}{Srednia2013}\PY{o}{\PYZlt{}\PYZhy{}}\PY{n+nf}{mean}\PY{p}{(}\PY{n+nf}{as.numeric}\PY{p}{(}\PY{n}{TowarRok}\PY{n+nf}{[c}\PY{p}{(}\PY{l+m}{8}\PY{p}{)}\PY{p}{,}\PY{n+nf}{c}\PY{p}{(}\PY{l+m}{1}\PY{o}{:}\PY{l+m}{10}\PY{p}{)}\PY{n}{]}\PY{p}{)}\PY{p}{,}\PY{n}{na.rm}\PY{o}{=}\PY{k+kc}{TRUE}\PY{p}{)}\PY{p}{,}

  \PY{n}{Srednia2014}\PY{o}{\PYZlt{}\PYZhy{}}\PY{n+nf}{mean}\PY{p}{(}\PY{n+nf}{as.numeric}\PY{p}{(}\PY{n}{TowarRok}\PY{n+nf}{[c}\PY{p}{(}\PY{l+m}{9}\PY{p}{)}\PY{p}{,}\PY{n+nf}{c}\PY{p}{(}\PY{l+m}{1}\PY{o}{:}\PY{l+m}{10}\PY{p}{)}\PY{n}{]}\PY{p}{)}\PY{p}{,}\PY{n}{na.rm}\PY{o}{=}\PY{k+kc}{TRUE}\PY{p}{)}\PY{p}{,}

  \PY{n}{Srednia2015}\PY{o}{\PYZlt{}\PYZhy{}}\PY{n+nf}{mean}\PY{p}{(}\PY{n+nf}{as.numeric}\PY{p}{(}\PY{n}{TowarRok}\PY{n+nf}{[c}\PY{p}{(}\PY{l+m}{10}\PY{p}{)}\PY{p}{,}\PY{n+nf}{c}\PY{p}{(}\PY{l+m}{1}\PY{o}{:}\PY{l+m}{10}\PY{p}{)}\PY{n}{]}\PY{p}{)}\PY{p}{,}\PY{n}{na.rm}\PY{o}{=}\PY{k+kc}{TRUE}\PY{p}{)}\PY{p}{,}

  \PY{n}{Srednia2016}\PY{o}{\PYZlt{}\PYZhy{}}\PY{n+nf}{mean}\PY{p}{(}\PY{n+nf}{as.numeric}\PY{p}{(}\PY{n}{TowarRok}\PY{n+nf}{[c}\PY{p}{(}\PY{l+m}{11}\PY{p}{)}\PY{p}{,}\PY{n+nf}{c}\PY{p}{(}\PY{l+m}{1}\PY{o}{:}\PY{l+m}{10}\PY{p}{)}\PY{n}{]}\PY{p}{)}\PY{p}{,}\PY{n}{na.rm}\PY{o}{=}\PY{k+kc}{TRUE}\PY{p}{)}\PY{p}{,}

  \PY{n}{Srednia2017}\PY{o}{\PYZlt{}\PYZhy{}}\PY{n+nf}{mean}\PY{p}{(}\PY{n+nf}{as.numeric}\PY{p}{(}\PY{n}{TowarRok}\PY{n+nf}{[c}\PY{p}{(}\PY{l+m}{12}\PY{p}{)}\PY{p}{,}\PY{n+nf}{c}\PY{p}{(}\PY{l+m}{1}\PY{o}{:}\PY{l+m}{10}\PY{p}{)}\PY{n}{]}\PY{p}{)}\PY{p}{,}\PY{n}{na.rm}\PY{o}{=}\PY{k+kc}{TRUE}\PY{p}{)}\PY{p}{,}

  \PY{n}{Srednia2018}\PY{o}{\PYZlt{}\PYZhy{}}\PY{n+nf}{mean}\PY{p}{(}\PY{n+nf}{as.numeric}\PY{p}{(}\PY{n}{TowarRok}\PY{n+nf}{[c}\PY{p}{(}\PY{l+m}{13}\PY{p}{)}\PY{p}{,}\PY{n+nf}{c}\PY{p}{(}\PY{l+m}{1}\PY{o}{:}\PY{l+m}{10}\PY{p}{)}\PY{n}{]}\PY{p}{)}\PY{p}{,}\PY{n}{na.rm}\PY{o}{=}\PY{k+kc}{TRUE}\PY{p}{)}\PY{p}{,}

  \PY{n}{Srednia2019}\PY{o}{\PYZlt{}\PYZhy{}}\PY{n+nf}{mean}\PY{p}{(}\PY{n+nf}{as.numeric}\PY{p}{(}\PY{n}{TowarRok}\PY{n+nf}{[c}\PY{p}{(}\PY{l+m}{14}\PY{p}{)}\PY{p}{,}\PY{n+nf}{c}\PY{p}{(}\PY{l+m}{1}\PY{o}{:}\PY{l+m}{10}\PY{p}{)}\PY{n}{]}\PY{p}{)}\PY{p}{,}\PY{n}{na.rm}\PY{o}{=}\PY{k+kc}{TRUE}\PY{p}{)}
\PY{p}{)}
\end{Verbatim}
\end{tcolorbox}

    \hypertarget{na-potrzeby-wykresuxf3w-i-ramek-tworzux119-listux119-z-latami-od-2006-do-2019}{%
\subparagraph{Na potrzeby wykresów i ramek, tworzę listę z latami od
2006 do
2019}\label{na-potrzeby-wykresuxf3w-i-ramek-tworzux119-listux119-z-latami-od-2006-do-2019}}

    \begin{tcolorbox}[breakable, size=fbox, boxrule=1pt, pad at break*=1mm,colback=cellbackground, colframe=cellborder]
\prompt{In}{incolor}{ }{\boxspacing}
\begin{Verbatim}[commandchars=\\\{\}]
\PY{n}{lata}\PY{o}{\PYZlt{}\PYZhy{}}\PY{n+nf}{c}\PY{p}{(}\PY{l+m}{2006}\PY{o}{:}\PY{l+m}{2019}\PY{p}{)}

\PY{n}{lista\PYZus{}srednich}\PY{o}{\PYZlt{}\PYZhy{}}\PY{n+nf}{data.frame}\PY{p}{(}
  \PY{n}{lata}\PY{o}{=}\PY{n+nf}{c}\PY{p}{(}\PY{n}{lata}\PY{p}{)}\PY{p}{,}
  \PY{n}{ceny}\PY{o}{=}\PY{n+nf}{c}\PY{p}{(}\PY{n}{lista\PYZus{}srednich\PYZus{}cen}\PY{p}{)}
\PY{p}{)}     
\PY{n+nf}{print}\PY{p}{(}\PY{n}{lista\PYZus{}srednich}\PY{p}{)}
\end{Verbatim}
\end{tcolorbox}

    \hypertarget{zmieniam-kolumny-z-wierszami-aby-uux142atwiux107-robienie-wykresuxf3w}{%
\subparagraph{Zmieniam kolumny z wierszami, aby ułatwić robienie
wykresów}\label{zmieniam-kolumny-z-wierszami-aby-uux142atwiux107-robienie-wykresuxf3w}}

    \begin{tcolorbox}[breakable, size=fbox, boxrule=1pt, pad at break*=1mm,colback=cellbackground, colframe=cellborder]
\prompt{In}{incolor}{ }{\boxspacing}
\begin{Verbatim}[commandchars=\\\{\}]
\PY{n}{Roczne\PYZus{}ceny}\PY{o}{\PYZlt{}\PYZhy{}}\PY{n+nf}{t}\PY{p}{(}\PY{n}{wojewodztwa\PYZus{}ceny}\PY{p}{)}
\PY{n}{Roczne\PYZus{}ceny}\PY{o}{\PYZlt{}\PYZhy{}}\PY{n+nf}{as.data.frame}\PY{p}{(}\PY{n}{Roczne\PYZus{}ceny}\PY{p}{)}
\PY{n}{nazwy}\PY{o}{\PYZlt{}\PYZhy{}}\PY{n+nf}{row.names}\PY{p}{(}\PY{n}{Roczne\PYZus{}ceny}\PY{p}{)}
\end{Verbatim}
\end{tcolorbox}

    \hypertarget{dzieki-tej-czux119ux15bci-bux119dux119-muxf3gux142-wykazaux107-gdzie-i-kiedy-byux142o-najdroux17cej-a-gdzie-i-kiedy-najtaniej}{%
\subparagraph{Dzieki tej części, będę mógł wykazać gdzie i kiedy było
najdrożej, a gdzie i kiedy
najtaniej}\label{dzieki-tej-czux119ux15bci-bux119dux119-muxf3gux142-wykazaux107-gdzie-i-kiedy-byux142o-najdroux17cej-a-gdzie-i-kiedy-najtaniej}}

    \begin{tcolorbox}[breakable, size=fbox, boxrule=1pt, pad at break*=1mm,colback=cellbackground, colframe=cellborder]
\prompt{In}{incolor}{ }{\boxspacing}
\begin{Verbatim}[commandchars=\\\{\}]
\PY{n}{nazwy\PYZus{}do\PYZus{}listy}\PY{o}{\PYZlt{}\PYZhy{}}\PY{n+nf}{rep}\PY{p}{(}\PY{n}{nazwy}\PY{p}{,}\PY{n}{each}\PY{o}{=}\PY{l+m}{14}\PY{p}{)}
\PY{n}{minmax}\PY{o}{\PYZlt{}\PYZhy{}}\PY{n+nf}{data.frame}\PY{p}{(}
  \PY{n}{wojewodztwa}\PY{o}{=}\PY{n+nf}{c}\PY{p}{(}\PY{n}{nazwy\PYZus{}do\PYZus{}listy}\PY{p}{)}\PY{p}{,}
  \PY{n}{wartosci}\PY{o}{=}\PY{n+nf}{c}\PY{p}{(}\PY{n}{wojewodztwa\PYZus{}ceny}\PY{o}{\PYZdl{}}\PY{n}{DOLNOŚLĄSKIE}\PY{p}{,}\PY{n}{wojewodztwa\PYZus{}ceny}\PY{o}{\PYZdl{}}\PY{n}{KUJAWSKO.POMORSKIE}\PY{p}{,}\PY{n}{wojewodztwa\PYZus{}ceny}\PY{o}{\PYZdl{}}\PY{n}{LUBELSKIE}\PY{p}{,}\PY{n}{wojewodztwa\PYZus{}ceny}\PY{o}{\PYZdl{}}\PY{n}{LUBUSKIE}\PY{p}{,}
             \PY{n}{wojewodztwa\PYZus{}ceny}\PY{o}{\PYZdl{}}ŁÓ\PY{n}{DZKIE}\PY{p}{,} \PY{n}{wojewodztwa\PYZus{}ceny}\PY{o}{\PYZdl{}}\PY{n}{MAŁOPOLSKIE}\PY{p}{,}\PY{n}{wojewodztwa\PYZus{}ceny}\PY{o}{\PYZdl{}}\PY{n}{MAZOWIECKIE}\PY{p}{,}\PY{n}{wojewodztwa\PYZus{}ceny}\PY{o}{\PYZdl{}}\PY{n}{OPOLSKIE}\PY{p}{,}
             \PY{n}{wojewodztwa\PYZus{}ceny}\PY{o}{\PYZdl{}}\PY{n}{PODKARPACKIE}\PY{p}{,}\PY{n}{wojewodztwa\PYZus{}ceny}\PY{o}{\PYZdl{}}\PY{n}{PODLASKIE}\PY{p}{,}\PY{n}{wojewodztwa\PYZus{}ceny}\PY{o}{\PYZdl{}}\PY{n}{POMORSKIE}\PY{p}{,}\PY{n}{wojewodztwa\PYZus{}ceny}\PY{o}{\PYZdl{}}Ś\PY{n}{LĄSKIE}\PY{p}{,}
             \PY{n}{wojewodztwa\PYZus{}ceny}\PY{o}{\PYZdl{}}Ś\PY{n}{WIĘTOKRZYSKIE}\PY{p}{,}\PY{n}{wojewodztwa\PYZus{}ceny}\PY{o}{\PYZdl{}}\PY{n}{WARMIŃSKO.MAZURSKIE}\PY{p}{,}\PY{n}{wojewodztwa\PYZus{}ceny}\PY{o}{\PYZdl{}}\PY{n}{WIELKOPOLSKIE}\PY{p}{,}
             \PY{n}{wojewodztwa\PYZus{}ceny}\PY{o}{\PYZdl{}}\PY{n}{ZACHODNIOPOMORSKIE}\PY{p}{)}\PY{p}{,}
  \PY{n}{lata}\PY{o}{=}\PY{n+nf}{c}\PY{p}{(}\PY{n}{lata}\PY{p}{)}
\PY{p}{)}
\PY{n}{DT}\PY{o}{\PYZlt{}\PYZhy{}}\PY{n+nf}{data.table}\PY{p}{(}\PY{n}{minmax}\PY{p}{)}
\end{Verbatim}
\end{tcolorbox}

    \hypertarget{czas-na-podsumowanie}{%
\subparagraph{CZAS NA PODSUMOWANIE}\label{czas-na-podsumowanie}}

\hypertarget{ponizej-zalaczam-plik-ze-statystcznym-podsumowaniem-kazdego-wojewodztwa}{%
\subparagraph{Ponizej zalaczam plik ze statystcznym podsumowaniem
kazdego
wojewodztwa}\label{ponizej-zalaczam-plik-ze-statystcznym-podsumowaniem-kazdego-wojewodztwa}}

    \begin{tcolorbox}[breakable, size=fbox, boxrule=1pt, pad at break*=1mm,colback=cellbackground, colframe=cellborder]
\prompt{In}{incolor}{ }{\boxspacing}
\begin{Verbatim}[commandchars=\\\{\}]
\PY{n}{WC}\PY{o}{\PYZlt{}\PYZhy{}}\PY{n+nf}{summary}\PY{p}{(}\PY{n}{wojewodztwa\PYZus{}ceny}\PY{p}{)}
\PY{n+nf}{print}\PY{p}{(}\PY{n}{WC}\PY{p}{)}
\PY{n+nf}{write.xlsx}\PY{p}{(}\PY{n+nf}{as.matrix}\PY{p}{(}\PY{n}{WC}\PY{p}{)}\PY{p}{,}\PY{l+s}{\PYZdq{}}\PY{l+s}{wojewodztwa\PYZhy{}podsumowanie\PYZhy{}statystyczne.xlsx\PYZdq{}}\PY{p}{)}
\end{Verbatim}
\end{tcolorbox}

    \hypertarget{teraz-bux119dzie-lista-najniux17cszych-cen-wzglux119dem-13-lat-w-kaux17cdym-wojewuxf3dztwie}{%
\subparagraph{Teraz będzie lista najniższych cen względem 13 lat w
każdym
województwie}\label{teraz-bux119dzie-lista-najniux17cszych-cen-wzglux119dem-13-lat-w-kaux17cdym-wojewuxf3dztwie}}

    \begin{tcolorbox}[breakable, size=fbox, boxrule=1pt, pad at break*=1mm,colback=cellbackground, colframe=cellborder]
\prompt{In}{incolor}{ }{\boxspacing}
\begin{Verbatim}[commandchars=\\\{\}]
\PY{n}{najnizszeCeny}\PY{o}{\PYZlt{}\PYZhy{}}\PY{n+nf}{as.data.frame}\PY{p}{(}\PY{n}{DT}\PY{n}{[} \PY{p}{,} \PY{n}{.SD}\PY{n+nf}{[which.min}\PY{p}{(}\PY{n}{wartosci}\PY{p}{)}\PY{n}{]}\PY{p}{,} \PY{n}{by} \PY{o}{=} \PY{n}{wojewodztwa}\PY{n}{]}\PY{p}{)}
\PY{n}{najnizszeCeny}\PY{o}{\PYZlt{}\PYZhy{}}\PY{n}{najnizszeCeny}\PY{n+nf}{[order}\PY{p}{(}\PY{n}{najnizszeCeny}\PY{o}{\PYZdl{}}\PY{n}{wartosci}\PY{p}{)}\PY{p}{,}\PY{n}{]}
\PY{n+nf}{write.xlsx}\PY{p}{(}\PY{n}{najnizszeCeny}\PY{p}{,}\PY{l+s}{\PYZdq{}}\PY{l+s}{Najniższe ceny \PYZhy{} województwa.xlsx\PYZdq{}}\PY{p}{)}
\end{Verbatim}
\end{tcolorbox}

    \hypertarget{czas-sprawdziux107-jaka-wartoux15bux107-byux142a-najniux17cszux105}{%
\subparagraph{Czas sprawdzić jaka wartość była
najniższą}\label{czas-sprawdziux107-jaka-wartoux15bux107-byux142a-najniux17cszux105}}

    \begin{tcolorbox}[breakable, size=fbox, boxrule=1pt, pad at break*=1mm,colback=cellbackground, colframe=cellborder]
\prompt{In}{incolor}{ }{\boxspacing}
\begin{Verbatim}[commandchars=\\\{\}]
\PY{n}{Najnizsza\PYZus{}Cena\PYZus{}W\PYZus{}Polsce}\PY{o}{\PYZlt{}\PYZhy{}}\PY{n+nf}{subset}\PY{p}{(}\PY{n}{minmax}\PY{p}{,}\PY{n}{wartosci}\PY{o}{==}\PY{n+nf}{min}\PY{p}{(}\PY{n}{wartosci}\PY{p}{)}\PY{p}{)}
\PY{n+nf}{print}\PY{p}{(}\PY{n}{Najnizsza\PYZus{}Cena\PYZus{}W\PYZus{}Polsce}\PY{p}{)}
\end{Verbatim}
\end{tcolorbox}

    \hypertarget{wiemy-juux17c-ux17ce-w-ux15blux105sku-w-2006-roku-ux15brednia-cena-byux142a-najniux17csza.-jak-wypadux142a-reszta-wojewuxf3dztw-w-okresie-13-lat}{%
\subparagraph{Wiemy już, że w Śląsku w 2006 roku średnia cena była
najniższa. Jak wypadła reszta województw w okresie 13
lat?}\label{wiemy-juux17c-ux17ce-w-ux15blux105sku-w-2006-roku-ux15brednia-cena-byux142a-najniux17csza.-jak-wypadux142a-reszta-wojewuxf3dztw-w-okresie-13-lat}}

    \begin{tcolorbox}[breakable, size=fbox, boxrule=1pt, pad at break*=1mm,colback=cellbackground, colframe=cellborder]
\prompt{In}{incolor}{ }{\boxspacing}
\begin{Verbatim}[commandchars=\\\{\}]
\PY{c+c1}{\PYZsh{}png(file=\PYZdq{}Wykres najnizszych cen.png\PYZdq{},width=700,height=500)}
\PY{n+nf}{ggplot}\PY{p}{(}\PY{n}{najnizszeCeny}\PY{p}{,}\PY{n+nf}{aes}\PY{p}{(}\PY{n}{x}\PY{o}{=}\PY{n+nf}{reorder}\PY{p}{(}\PY{n}{wojewodztwa}\PY{p}{,}\PY{o}{\PYZhy{}}\PY{n}{wartosci}\PY{p}{)}\PY{p}{,} \PY{n}{y}\PY{o}{=}\PY{n}{wartosci}\PY{p}{)}\PY{p}{)} \PY{o}{+}
  \PY{n+nf}{geom\PYZus{}bar}\PY{p}{(}\PY{n}{stat} \PY{o}{=} \PY{l+s}{\PYZdq{}}\PY{l+s}{identity\PYZdq{}}\PY{p}{,} \PY{n}{width}\PY{o}{=}\PY{l+m}{0.5}\PY{p}{,}\PY{n}{fill}\PY{o}{=}\PY{l+s}{\PYZdq{}}\PY{l+s}{deepskyblue1\PYZdq{}}\PY{p}{)}\PY{o}{+}
  \PY{n+nf}{geom\PYZus{}text}\PY{p}{(}\PY{n+nf}{aes}\PY{p}{(}\PY{n}{label}\PY{o}{=}\PY{n+nf}{sprintf}\PY{p}{(}\PY{l+s}{\PYZdq{}}\PY{l+s}{\PYZpc{}0.2f\PYZdq{}}\PY{p}{,} \PY{n+nf}{round}\PY{p}{(}\PY{n}{wartosci}\PY{p}{,} \PY{n}{digits} \PY{o}{=} \PY{l+m}{2}\PY{p}{)}\PY{p}{)}\PY{p}{)}\PY{p}{,} \PY{n}{vjust}\PY{o}{=}\PY{l+m}{0.5}\PY{p}{,} \PY{n}{color}\PY{o}{=}\PY{l+s}{\PYZdq{}}\PY{l+s}{white\PYZdq{}}\PY{p}{,} \PY{n}{size}\PY{o}{=}\PY{l+m}{5}\PY{p}{,}\PY{n}{position} \PY{o}{=} \PY{n+nf}{position\PYZus{}stack}\PY{p}{(}\PY{n}{vjust} \PY{o}{=} \PY{l+m}{0.1}\PY{p}{)}\PY{p}{)}\PY{o}{+}
  \PY{n+nf}{geom\PYZus{}text}\PY{p}{(}\PY{n+nf}{aes}\PY{p}{(}\PY{n}{label}\PY{o}{=}\PY{n}{lata}\PY{p}{)}\PY{p}{,}\PY{n}{vvjust}\PY{o}{=}\PY{l+m}{0.5}\PY{p}{,} \PY{n}{color}\PY{o}{=}\PY{l+s}{\PYZdq{}}\PY{l+s}{black\PYZdq{}}\PY{p}{,} \PY{n}{size}\PY{o}{=}\PY{l+m}{5}\PY{p}{,}\PY{n}{position} \PY{o}{=} \PY{n+nf}{position\PYZus{}stack}\PY{p}{(}\PY{n}{vjust} \PY{o}{=} \PY{l+m}{0.5}\PY{p}{)}\PY{p}{)}\PY{o}{+}
  \PY{n+nf}{coord\PYZus{}flip}\PY{p}{(}\PY{p}{)}\PY{o}{+}
  \PY{n+nf}{ggtitle}\PY{p}{(}\PY{l+s}{\PYZdq{}}\PY{l+s}{Najniższe ceny towarów w województwie w okresie 2006\PYZhy{}2019\PYZdq{}}\PY{p}{)}\PY{o}{+}
  \PY{n+nf}{theme\PYZus{}minimal}\PY{p}{(}\PY{p}{)}
\PY{c+c1}{\PYZsh{}dev.off()}
\end{Verbatim}
\end{tcolorbox}

    \hypertarget{teraz-lista-najwyux17cszych-cen}{%
\subparagraph{Teraz lista najwyższych
cen}\label{teraz-lista-najwyux17cszych-cen}}

    \begin{tcolorbox}[breakable, size=fbox, boxrule=1pt, pad at break*=1mm,colback=cellbackground, colframe=cellborder]
\prompt{In}{incolor}{ }{\boxspacing}
\begin{Verbatim}[commandchars=\\\{\}]
\PY{n}{najwyzszeCeny}\PY{o}{\PYZlt{}\PYZhy{}}\PY{n+nf}{as.data.frame}\PY{p}{(}\PY{n}{DT}\PY{n}{[} \PY{p}{,} \PY{n}{.SD}\PY{n+nf}{[which.max}\PY{p}{(}\PY{n}{wartosci}\PY{p}{)}\PY{n}{]}\PY{p}{,} \PY{n}{by} \PY{o}{=} \PY{n}{wojewodztwa}\PY{n}{]}\PY{p}{)}
\PY{n}{najwyzszeCeny}\PY{o}{\PYZlt{}\PYZhy{}}\PY{n}{najwyzszeCeny}\PY{n+nf}{[order}\PY{p}{(}\PY{n}{najwyzszeCeny}\PY{o}{\PYZdl{}}\PY{n}{wartosci}\PY{p}{)}\PY{p}{,}\PY{n}{]}
\end{Verbatim}
\end{tcolorbox}

    \hypertarget{czas-sprawdzic-najwyzsza-wartosc-w-okresie-13-lat}{%
\subparagraph{Czas sprawdzic najwyzsza wartosc w okresie 13
lat}\label{czas-sprawdzic-najwyzsza-wartosc-w-okresie-13-lat}}

    \begin{tcolorbox}[breakable, size=fbox, boxrule=1pt, pad at break*=1mm,colback=cellbackground, colframe=cellborder]
\prompt{In}{incolor}{ }{\boxspacing}
\begin{Verbatim}[commandchars=\\\{\}]
\PY{n+nf}{print}\PY{p}{(}\PY{n+nf}{subset}\PY{p}{(}\PY{n}{minmax}\PY{p}{,}\PY{n}{wartosci}\PY{o}{==}\PY{n+nf}{max}\PY{p}{(}\PY{n}{wartosci}\PY{p}{)}\PY{p}{)}\PY{p}{)}
\end{Verbatim}
\end{tcolorbox}

    \hypertarget{a-terazw-wykres-przedstawiajux105cy-wartoux15bci-najwiux119kszych-cen-w-okresie-2006-2019-na-obszarze-16-wojewuxf3dztw}{%
\subparagraph{A terazw wykres przedstawiający wartości największych cen
w okresie 2006-2019 na obszarze 16
województw}\label{a-terazw-wykres-przedstawiajux105cy-wartoux15bci-najwiux119kszych-cen-w-okresie-2006-2019-na-obszarze-16-wojewuxf3dztw}}

    \begin{tcolorbox}[breakable, size=fbox, boxrule=1pt, pad at break*=1mm,colback=cellbackground, colframe=cellborder]
\prompt{In}{incolor}{ }{\boxspacing}
\begin{Verbatim}[commandchars=\\\{\}]
\PY{c+c1}{\PYZsh{}pdf(file=\PYZdq{}Wykres najwyzszych cen.pdf\PYZdq{},width=700,height=500)}
\PY{n+nf}{ggplot}\PY{p}{(}\PY{n}{najwyzszeCeny}\PY{p}{,}\PY{n+nf}{aes}\PY{p}{(}\PY{n}{x}\PY{o}{=}\PY{n+nf}{reorder}\PY{p}{(}\PY{n}{wojewodztwa}\PY{p}{,}\PY{n}{wartosci}\PY{p}{)}\PY{p}{,} \PY{n}{y}\PY{o}{=}\PY{n}{wartosci}\PY{p}{)}\PY{p}{)} \PY{o}{+}
  \PY{n+nf}{geom\PYZus{}bar}\PY{p}{(}\PY{n}{stat} \PY{o}{=} \PY{l+s}{\PYZdq{}}\PY{l+s}{identity\PYZdq{}}\PY{p}{,} \PY{n}{width}\PY{o}{=}\PY{l+m}{0.5}\PY{p}{,}\PY{n}{fill}\PY{o}{=}\PY{l+s}{\PYZdq{}}\PY{l+s}{\PYZsh{}E69F00\PYZdq{}}\PY{p}{)}\PY{o}{+}
  \PY{n+nf}{geom\PYZus{}text}\PY{p}{(}\PY{n+nf}{aes}\PY{p}{(}\PY{n}{label}\PY{o}{=}\PY{n+nf}{sprintf}\PY{p}{(}\PY{l+s}{\PYZdq{}}\PY{l+s}{\PYZpc{}0.2f\PYZdq{}}\PY{p}{,} \PY{n+nf}{round}\PY{p}{(}\PY{n}{wartosci}\PY{p}{,} \PY{n}{digits} \PY{o}{=} \PY{l+m}{2}\PY{p}{)}\PY{p}{)}\PY{p}{)}\PY{p}{,} \PY{n}{vjust}\PY{o}{=}\PY{l+m}{0.5}\PY{p}{,} \PY{n}{color}\PY{o}{=}\PY{l+s}{\PYZdq{}}\PY{l+s}{black\PYZdq{}}\PY{p}{,} \PY{n}{size}\PY{o}{=}\PY{l+m}{5}\PY{p}{,}\PY{n}{position} \PY{o}{=} \PY{n+nf}{position\PYZus{}stack}\PY{p}{(}\PY{n}{vjust} \PY{o}{=} \PY{l+m}{0.1}\PY{p}{)}\PY{p}{)}\PY{o}{+}
  \PY{n+nf}{geom\PYZus{}text}\PY{p}{(}\PY{n+nf}{aes}\PY{p}{(}\PY{n}{label}\PY{o}{=}\PY{n}{lata}\PY{p}{)}\PY{p}{,}\PY{n}{vvjust}\PY{o}{=}\PY{l+m}{0.5}\PY{p}{,} \PY{n}{color}\PY{o}{=}\PY{l+s}{\PYZdq{}}\PY{l+s}{black\PYZdq{}}\PY{p}{,} \PY{n}{size}\PY{o}{=}\PY{l+m}{5}\PY{p}{,}\PY{n}{position} \PY{o}{=} \PY{n+nf}{position\PYZus{}stack}\PY{p}{(}\PY{n}{vjust} \PY{o}{=} \PY{l+m}{0.5}\PY{p}{)}\PY{p}{)}\PY{o}{+}
  \PY{n+nf}{coord\PYZus{}flip}\PY{p}{(}\PY{p}{)}\PY{o}{+}
  \PY{n+nf}{ggtitle}\PY{p}{(}\PY{l+s}{\PYZdq{}}\PY{l+s}{Najwyzsze ceny towarów w województwie w okresie 2006\PYZhy{}2019\PYZdq{}}\PY{p}{)}\PY{o}{+}
  \PY{n+nf}{theme\PYZus{}minimal}\PY{p}{(}\PY{p}{)}
\PY{c+c1}{\PYZsh{}dev.off()}
\end{Verbatim}
\end{tcolorbox}

    \hypertarget{poniux17cszy-wykres-przedstawia-zmianux119-krajowych-cen-omawianych-produktuxf3w-i-usux142ug-w}{%
\subparagraph{Poniższy wykres przedstawia zmianę krajowych cen
omawianych produktów i usług
w}\label{poniux17cszy-wykres-przedstawia-zmianux119-krajowych-cen-omawianych-produktuxf3w-i-usux142ug-w}}

\hypertarget{okresie-2006-2019-na-obszarze-caux142ej-polski}{%
\subparagraph{okresie 2006-2019 na obszarze całej
Polski}\label{okresie-2006-2019-na-obszarze-caux142ej-polski}}

    \begin{tcolorbox}[breakable, size=fbox, boxrule=1pt, pad at break*=1mm,colback=cellbackground, colframe=cellborder]
\prompt{In}{incolor}{ }{\boxspacing}
\begin{Verbatim}[commandchars=\\\{\}]
\PY{c+c1}{\PYZsh{}png(file=\PYZdq{}Wykres cen w Polsce.png\PYZdq{},width=700,height=500)}
\PY{n+nf}{ggplot}\PY{p}{(}\PY{n}{lista\PYZus{}srednich}\PY{p}{,} \PY{n+nf}{aes}\PY{p}{(}\PY{n}{x}\PY{o}{=}\PY{n}{lata}\PY{p}{,} \PY{n}{y}\PY{o}{=}\PY{n}{ceny}\PY{p}{)}\PY{p}{)} \PY{o}{+}
  \PY{n+nf}{geom\PYZus{}line}\PY{p}{(} \PY{n}{color}\PY{o}{=}\PY{l+s}{\PYZdq{}}\PY{l+s}{grey\PYZdq{}}\PY{p}{)} \PY{o}{+}
  \PY{n+nf}{theme\PYZus{}ipsum\PYZus{}rc}\PY{p}{(}\PY{p}{)}\PY{o}{+}
  \PY{n+nf}{geom\PYZus{}point}\PY{p}{(}\PY{n}{shape}\PY{o}{=}\PY{l+m}{4}\PY{p}{,} \PY{n}{color}\PY{o}{=}\PY{l+s}{\PYZdq{}}\PY{l+s}{black\PYZdq{}}\PY{p}{,} \PY{n}{fill}\PY{o}{=}\PY{l+s}{\PYZdq{}}\PY{l+s}{\PYZsh{}69b3a2\PYZdq{}}\PY{p}{,} \PY{n}{size}\PY{o}{=}\PY{l+m}{6}\PY{p}{)} \PY{o}{+}
  \PY{n+nf}{ggtitle}\PY{p}{(}\PY{l+s}{\PYZdq{}}\PY{l+s}{Zmiana srednich cen w Polsce w latach 2006\PYZhy{}2019\PYZdq{}}\PY{p}{)}\PY{o}{+}
  \PY{n+nf}{scale\PYZus{}x\PYZus{}continuous}\PY{p}{(}\PY{n}{breaks} \PY{o}{=} \PY{n+nf}{round}\PY{p}{(}\PY{n+nf}{seq}\PY{p}{(}\PY{n+nf}{min}\PY{p}{(}\PY{n}{lista\PYZus{}srednich}\PY{o}{\PYZdl{}}\PY{n}{lata}\PY{p}{)}\PY{p}{,} \PY{n+nf}{max}\PY{p}{(}\PY{n}{lista\PYZus{}srednich}\PY{o}{\PYZdl{}}\PY{n}{lata}\PY{p}{)}\PY{p}{,} \PY{n}{by} \PY{o}{=} \PY{l+m}{1}\PY{p}{)}\PY{p}{,}\PY{l+m}{1}\PY{p}{)}\PY{p}{)}\PY{o}{+}
  \PY{n+nf}{scale\PYZus{}y\PYZus{}continuous}\PY{p}{(}\PY{n}{breaks} \PY{o}{=} \PY{n+nf}{round}\PY{p}{(}\PY{n+nf}{seq}\PY{p}{(}\PY{n+nf}{min}\PY{p}{(}\PY{n}{lista\PYZus{}srednich}\PY{o}{\PYZdl{}}\PY{n}{ceny}\PY{p}{)}\PY{p}{,} \PY{n+nf}{max}\PY{p}{(}\PY{n}{lista\PYZus{}srednich}\PY{o}{\PYZdl{}}\PY{n}{ceny}\PY{p}{)}\PY{p}{,} \PY{n}{by} \PY{o}{=} \PY{l+m}{5}\PY{p}{)}\PY{p}{,}\PY{l+m}{1}\PY{p}{)}\PY{p}{)} 
\PY{c+c1}{\PYZsh{}dev.off()}
\end{Verbatim}
\end{tcolorbox}

    \hypertarget{zgodnie-z-powyux17cszym-wykresem-jeden-z-najwiux119kszych-wzrost-cen-moux17cna-zaobserwowaux107-od-2016-roku}{%
\subparagraph{Zgodnie z powyższym wykresem, jeden z największych wzrost
cen można zaobserwować od 2016
roku}\label{zgodnie-z-powyux17cszym-wykresem-jeden-z-najwiux119kszych-wzrost-cen-moux17cna-zaobserwowaux107-od-2016-roku}}

\hypertarget{wiux105ux17ce-siux119-to-z-wdroux17ceniem-wielu-programuxf3w-socjalnych-w-tym-500}{%
\subparagraph{Wiąże się to z wdrożeniem wielu programów socjalnych, w
tym
500+}\label{wiux105ux17ce-siux119-to-z-wdroux17ceniem-wielu-programuxf3w-socjalnych-w-tym-500}}

\hypertarget{poniux17cej-natomiast-znajduje-siux119-wartoux15bux107-korelacji-miux119dzy-minialnym-wynagrodzeniem-w-danym-roku-a-cenux105-towaruxf3w}{%
\subparagraph{Poniżej natomiast znajduje się wartość korelacji między
minialnym wynagrodzeniem w danym roku a ceną
towarów}\label{poniux17cej-natomiast-znajduje-siux119-wartoux15bux107-korelacji-miux119dzy-minialnym-wynagrodzeniem-w-danym-roku-a-cenux105-towaruxf3w}}

    \begin{tcolorbox}[breakable, size=fbox, boxrule=1pt, pad at break*=1mm,colback=cellbackground, colframe=cellborder]
\prompt{In}{incolor}{ }{\boxspacing}
\begin{Verbatim}[commandchars=\\\{\}]
\PY{c+c1}{\PYZsh{}korelacja między cenami a średnią wynagrodzenia}
\PY{n}{srednie\PYZus{}wynagrodzenie}\PY{o}{\PYZlt{}\PYZhy{}}\PY{n+nf}{c}\PY{p}{(}\PY{l+m}{899}\PY{p}{,}\PY{l+m}{936}\PY{p}{,}\PY{l+m}{1126}\PY{p}{,}\PY{l+m}{1276}\PY{p}{,}\PY{l+m}{1317}\PY{p}{,}
                         \PY{l+m}{1386}\PY{p}{,}\PY{l+m}{1500}\PY{p}{,}\PY{l+m}{1600}\PY{p}{,}\PY{l+m}{1680} \PY{p}{,}\PY{l+m}{1750}\PY{p}{,}
                         \PY{l+m}{1850}\PY{p}{,}\PY{l+m}{2000}\PY{p}{,}\PY{l+m}{2100}\PY{p}{,}\PY{l+m}{2250}\PY{p}{)}
\PY{n}{ceny\PYZus{}a\PYZus{}wynagrodzenie}\PY{o}{\PYZlt{}\PYZhy{}}\PY{n+nf}{data.frame}\PY{p}{(}
  \PY{n}{lata}\PY{o}{=}\PY{n+nf}{c}\PY{p}{(}\PY{n}{lata}\PY{p}{)}\PY{p}{,}
  \PY{n}{wynagrodzenie}\PY{o}{=}\PY{n+nf}{c}\PY{p}{(}\PY{n}{srednie\PYZus{}wynagrodzenie}\PY{p}{)}\PY{p}{,}
  \PY{n}{ceny}\PY{o}{=}\PY{n+nf}{c}\PY{p}{(}\PY{n}{lista\PYZus{}srednich\PYZus{}cen}\PY{p}{)}
\PY{p}{)}
\PY{n}{korelacja}\PY{o}{\PYZlt{}\PYZhy{}}\PY{n+nf}{cor}\PY{p}{(}\PY{n}{ceny\PYZus{}a\PYZus{}wynagrodzenie}\PY{o}{\PYZdl{}}\PY{n}{ceny}\PY{p}{,}\PY{n}{ceny\PYZus{}a\PYZus{}wynagrodzenie}\PY{o}{\PYZdl{}}\PY{n}{wynagrodzenie}\PY{p}{)}
\PY{n+nf}{print}\PY{p}{(}\PY{n}{korelacja}\PY{p}{)}
\end{Verbatim}
\end{tcolorbox}

    \hypertarget{korelacja-jest-bardzo-wysoka-wiux119c-sprawdzux119-teraz-regresjux119-liniowux105}{%
\subparagraph{Korelacja jest bardzo wysoka, więc sprawdzę teraz regresję
liniową}\label{korelacja-jest-bardzo-wysoka-wiux119c-sprawdzux119-teraz-regresjux119-liniowux105}}

    \begin{tcolorbox}[breakable, size=fbox, boxrule=1pt, pad at break*=1mm,colback=cellbackground, colframe=cellborder]
\prompt{In}{incolor}{ }{\boxspacing}
\begin{Verbatim}[commandchars=\\\{\}]
\PY{n}{regresja}\PY{o}{\PYZlt{}\PYZhy{}}\PY{n+nf}{lm}\PY{p}{(}\PY{n}{ceny\PYZus{}a\PYZus{}wynagrodzenie}\PY{o}{\PYZdl{}}\PY{n}{ceny}\PY{o}{\PYZti{}}\PY{n}{ceny\PYZus{}a\PYZus{}wynagrodzenie}\PY{o}{\PYZdl{}}\PY{n}{wynagrodzenie}\PY{p}{)}

\PY{n+nf}{plot}\PY{p}{(}\PY{n}{ceny\PYZus{}a\PYZus{}wynagrodzenie}\PY{o}{\PYZdl{}}\PY{n}{wynagrodzenie}\PY{p}{,}\PY{n}{ceny\PYZus{}a\PYZus{}wynagrodzenie}\PY{o}{\PYZdl{}}\PY{n}{ceny}\PY{p}{,} \PY{n}{col}\PY{o}{=}\PY{l+s}{\PYZdq{}}\PY{l+s}{blue\PYZdq{}}\PY{p}{,} \PY{n}{main}\PY{o}{=}\PY{l+s}{\PYZdq{}}\PY{l+s}{Regresja wynagrodzenie\PYZti{}ceny\PYZdq{}}\PY{p}{,}
     \PY{n+nf}{abline}\PY{p}{(}\PY{n}{regresja}\PY{p}{)}\PY{p}{,}\PY{n}{cex}\PY{o}{=}\PY{l+m}{1.3}\PY{p}{,} \PY{n}{pch}\PY{o}{=} \PY{l+m}{16}\PY{p}{,} \PY{n}{xlab}\PY{o}{=} \PY{l+s}{\PYZdq{}}\PY{l+s}{minimalne wynagrodzenie\PYZdq{}}\PY{p}{,} \PY{n}{ylab}\PY{o}{=}\PY{l+s}{\PYZdq{}}\PY{l+s}{ceny\PYZdq{}}\PY{p}{)}
\end{Verbatim}
\end{tcolorbox}

    \hypertarget{jasno-z-tego-wynika-ux17ce-wraz-z-nagux142ym-wzrostem-pux142ac-i-programuxf3w-socjalnych-wzrasta-cena-produktuxf3w-a-tym-samym-wzrasta-inflacja}{%
\subparagraph{Jasno z tego wynika, że wraz z nagłym wzrostem płac i
programów socjalnych, wzrasta cena produktów, a tym samym wzrasta
inflacja}\label{jasno-z-tego-wynika-ux17ce-wraz-z-nagux142ym-wzrostem-pux142ac-i-programuxf3w-socjalnych-wzrasta-cena-produktuxf3w-a-tym-samym-wzrasta-inflacja}}

    \hypertarget{wnioski---dziux119ki-odpowiedniej-przeruxf3bce-danych-moux17cna-uzyskaux107-odpowiedzi-na-wiele-pytaux144---od-ux15bredniej-ceny-produktuxf3w-po-wzrost-miesiux119cznej-ceny-cytryny}{%
\section{Wnioski - dzięki odpowiedniej przeróbce danych, można uzyskać
odpowiedzi na wiele pytań - od średniej ceny produktów, po wzrost
miesięcznej ceny
cytryny}\label{wnioski---dziux119ki-odpowiedniej-przeruxf3bce-danych-moux17cna-uzyskaux107-odpowiedzi-na-wiele-pytaux144---od-ux15bredniej-ceny-produktuxf3w-po-wzrost-miesiux119cznej-ceny-cytryny}}

    \hypertarget{jednak-najwaux17cniejszym-wnioskiem-jest-fakt-ux17ce-ceny-towaruxf3w-rosnux105-zarobki-ruxf3wnieux17c}{%
\subsection{Jednak najważniejszym wnioskiem jest fakt, że ceny towarów
rosną, zarobki
również}\label{jednak-najwaux17cniejszym-wnioskiem-jest-fakt-ux17ce-ceny-towaruxf3w-rosnux105-zarobki-ruxf3wnieux17c}}

\hypertarget{w-niedalekiej-przyszux142oux15bci-moux17ce-to-doprowadziux107-do-duux17cej-inflacji-a-ceny-wruxf3cux105-do-wartoux15bci-sprzed-denominacji-z-1995-roku}{%
\subsection{W niedalekiej przyszłości może to doprowadzić do dużej
inflacji, a ceny wrócą do wartości sprzed denominacji z 1995
roku}\label{w-niedalekiej-przyszux142oux15bci-moux17ce-to-doprowadziux107-do-duux17cej-inflacji-a-ceny-wruxf3cux105-do-wartoux15bci-sprzed-denominacji-z-1995-roku}}

    \hypertarget{brak-danych-niektuxf3rych-cen-rozwiux105zaux142em-innym-plikiem-gus-owski-na-ktuxf3rym-te-dane-siux119-znajdowaux142y.}{%
\subparagraph{Brak danych niektórych cen rozwiązałem innym plikiem
GUS-owski, na którym te dane się
znajdowały.}\label{brak-danych-niektuxf3rych-cen-rozwiux105zaux142em-innym-plikiem-gus-owski-na-ktuxf3rym-te-dane-siux119-znajdowaux142y.}}

\hypertarget{innym-rozwiux105zaniem-byux142aby-predykcja-przyszux142ych-cen-na-bazie-dostux119pnych}{%
\subparagraph{Innym rozwiązaniem byłaby predykcja przyszłych cen na
bazie
dostępnych}\label{innym-rozwiux105zaniem-byux142aby-predykcja-przyszux142ych-cen-na-bazie-dostux119pnych}}

    \hypertarget{teraz-czas-na-pytanie---co-jeszcze-moux17cna-zrobiux107-z-tymi-danymi}{%
\subparagraph{Teraz czas na pytanie - co jeszcze można zrobić z tymi
danymi?}\label{teraz-czas-na-pytanie---co-jeszcze-moux17cna-zrobiux107-z-tymi-danymi}}

    \hypertarget{przykux142ad-1}{%
\subsection{Przykład 1}\label{przykux142ad-1}}

    \hypertarget{moux17cna-zebraux107-dane-ux15bredniej-ceny-artykuux142uxf3w-spoux17cywczych-z-kaux17cdego-roku-dla-kaux17cdego-wojewuxf3dztwa}{%
\subparagraph{Można zebrać dane średniej ceny ARTYKUŁÓW SPOŻYWCZYCH z
każdego roku dla każdego
województwa}\label{moux17cna-zebraux107-dane-ux15bredniej-ceny-artykuux142uxf3w-spoux17cywczych-z-kaux17cdego-roku-dla-kaux17cdego-wojewuxf3dztwa}}

    \begin{tcolorbox}[breakable, size=fbox, boxrule=1pt, pad at break*=1mm,colback=cellbackground, colframe=cellborder]
\prompt{In}{incolor}{ }{\boxspacing}
\begin{Verbatim}[commandchars=\\\{\}]
\PY{n}{przyklad1}\PY{o}{\PYZlt{}\PYZhy{}}\PY{n+nf}{tapply}\PY{p}{(}\PY{n}{dane}\PY{o}{\PYZdl{}}\PY{n}{Wartosc}\PY{p}{,}\PY{n+nf}{list}\PY{p}{(}\PY{n}{dane}\PY{o}{\PYZdl{}}\PY{n}{Towar}\PY{p}{,}\PY{n}{dane}\PY{o}{\PYZdl{}}\PY{n}{Nazwa}\PY{p}{,}\PY{n}{dane}\PY{o}{\PYZdl{}}\PY{n}{Rok}\PY{p}{)}\PY{p}{,}\PY{n}{FUN}\PY{o}{=}\PY{n}{mean}\PY{p}{,}\PY{n}{na.rm}\PY{o}{=}\PY{k+kc}{TRUE}\PY{p}{)}
\PY{n}{przyklad1}\PY{o}{\PYZlt{}\PYZhy{}}\PY{n+nf}{data.frame}\PY{p}{(}\PY{n}{przyklad1}\PY{p}{,}\PY{n}{stringsAsFactors} \PY{o}{=} \PY{k+kc}{FALSE}\PY{p}{)} 
\PY{n}{przyklad1}\PY{o}{\PYZlt{}\PYZhy{}}\PY{n+nf}{data.frame}\PY{p}{(}\PY{n}{przyklad1}\PY{n}{[}\PY{o}{\PYZhy{}}\PY{n+nf}{c}\PY{p}{(}\PY{l+m}{7}\PY{p}{,}\PY{l+m}{9}\PY{p}{,}\PY{l+m}{10}\PY{p}{)}\PY{p}{,}\PY{n}{]}\PY{p}{,}\PY{n}{stringsAsFactors} \PY{o}{=} \PY{k+kc}{FALSE}\PY{p}{)} \PY{c+c1}{\PYZsh{}Tutaj usuwam węgiel oraz spodnie i pranie}

\PY{n}{listaProduktow}\PY{o}{\PYZlt{}\PYZhy{}}\PY{n+nf}{row.names}\PY{p}{(}\PY{n}{przyklad1}\PY{p}{)} \PY{c+c1}{\PYZsh{}Tutaj jest uniwersalna lista nazw}
\end{Verbatim}
\end{tcolorbox}

    \hypertarget{przykux142ad-zrobimy-dla-dolnego-ux15blux105ska-z-2006-roku}{%
\subparagraph{Przykład zrobimy dla Dolnego Śląska z 2006
roku}\label{przykux142ad-zrobimy-dla-dolnego-ux15blux105ska-z-2006-roku}}

    \begin{tcolorbox}[breakable, size=fbox, boxrule=1pt, pad at break*=1mm,colback=cellbackground, colframe=cellborder]
\prompt{In}{incolor}{ }{\boxspacing}
\begin{Verbatim}[commandchars=\\\{\}]
\PY{n}{przyklad1\PYZus{}Dolnys2006}\PY{o}{\PYZlt{}\PYZhy{}}\PY{n+nf}{data.frame}\PY{p}{(}\PY{n}{produkt}\PY{o}{=}\PY{n+nf}{c}\PY{p}{(}\PY{n}{listaProduktow}\PY{p}{)}\PY{p}{,}\PY{n}{ceny}\PY{o}{=}\PY{n}{przyklad1}\PY{o}{\PYZdl{}}\PY{n}{DOLNOŚLĄSKIE.2006}\PY{p}{,}\PY{n}{stringsAsFactors} \PY{o}{=} \PY{k+kc}{FALSE}\PY{p}{)}
\PY{n+nf}{ggplot}\PY{p}{(}\PY{n}{przyklad1\PYZus{}Dolnys2006}\PY{p}{,}\PY{n+nf}{aes}\PY{p}{(}\PY{n}{x}\PY{o}{=}\PY{n}{produkt}\PY{p}{,} \PY{n}{y}\PY{o}{=}\PY{n}{ceny}\PY{p}{)}\PY{p}{)} \PY{o}{+}
     \PY{n+nf}{geom\PYZus{}bar}\PY{p}{(}\PY{n}{stat} \PY{o}{=} \PY{l+s}{\PYZdq{}}\PY{l+s}{identity\PYZdq{}}\PY{p}{,} \PY{n}{width}\PY{o}{=}\PY{l+m}{0.4}\PY{p}{,}\PY{n}{fill}\PY{o}{=}\PY{l+s}{\PYZdq{}}\PY{l+s}{red\PYZdq{}}\PY{p}{)}\PY{o}{+}
     \PY{n+nf}{geom\PYZus{}text}\PY{p}{(}\PY{n+nf}{aes}\PY{p}{(}\PY{n}{label}\PY{o}{=}\PY{n+nf}{sprintf}\PY{p}{(}\PY{l+s}{\PYZdq{}}\PY{l+s}{\PYZpc{}0.2f\PYZdq{}}\PY{p}{,} \PY{n+nf}{round}\PY{p}{(}\PY{n}{ceny}\PY{p}{,} \PY{n}{digits} \PY{o}{=} \PY{l+m}{2}\PY{p}{)}\PY{p}{)}\PY{p}{)}\PY{p}{,} \PY{n}{vjust}\PY{o}{=}\PY{l+m}{0.5}\PY{p}{,} \PY{n}{color}\PY{o}{=}\PY{l+s}{\PYZdq{}}\PY{l+s}{white\PYZdq{}}\PY{p}{,} \PY{n}{size}\PY{o}{=}\PY{l+m}{4.5}\PY{p}{,}\PY{n}{position} \PY{o}{=} \PY{n+nf}{position\PYZus{}stack}\PY{p}{(}\PY{n}{vjust} \PY{o}{=} \PY{l+m}{0.5}\PY{p}{)}\PY{p}{)}\PY{o}{+}
     \PY{n+nf}{coord\PYZus{}flip}\PY{p}{(}\PY{p}{)}\PY{o}{+}
     \PY{n+nf}{ggtitle}\PY{p}{(}\PY{l+s}{\PYZdq{}}\PY{l+s}{Srednie ceny produktow w Dolnyśląsk 2006\PYZdq{}}\PY{p}{)}\PY{o}{+}
     \PY{n+nf}{theme\PYZus{}minimal}\PY{p}{(}\PY{p}{)}
\end{Verbatim}
\end{tcolorbox}

    \hypertarget{przykux142ad-2}{%
\subsection{Przykład 2}\label{przykux142ad-2}}

    \hypertarget{teraz-spruxf3bujemy-okreux15bliux107-jak-zmieniaux142a-siux119-cena-towaru-w-skali-kraju-w-przeciux105gu-13-lat}{%
\subsubsection{Teraz spróbujemy określić jak zmieniała się cena towaru w
skali kraju w przeciągu 13
lat}\label{teraz-spruxf3bujemy-okreux15bliux107-jak-zmieniaux142a-siux119-cena-towaru-w-skali-kraju-w-przeciux105gu-13-lat}}

    \begin{tcolorbox}[breakable, size=fbox, boxrule=1pt, pad at break*=1mm,colback=cellbackground, colframe=cellborder]
\prompt{In}{incolor}{ }{\boxspacing}
\begin{Verbatim}[commandchars=\\\{\}]
\PY{n}{przyklad2}\PY{o}{\PYZlt{}\PYZhy{}}\PY{n+nf}{tapply}\PY{p}{(}\PY{n}{dane}\PY{o}{\PYZdl{}}\PY{n}{Wartosc}\PY{p}{,}\PY{n+nf}{list}\PY{p}{(}\PY{n}{dane}\PY{o}{\PYZdl{}}\PY{n}{Rok}\PY{p}{,}\PY{n}{dane}\PY{o}{\PYZdl{}}\PY{n}{Towar}\PY{p}{)}\PY{p}{,}\PY{n}{FUN}\PY{o}{=}\PY{n}{mean}\PY{p}{,}\PY{n}{na.rm}\PY{o}{=}\PY{k+kc}{TRUE}\PY{p}{)}
\PY{n}{przyklad2}\PY{o}{\PYZlt{}\PYZhy{}}\PY{n+nf}{data.frame}\PY{p}{(}\PY{n}{przyklad2}\PY{p}{)}
\PY{n+nf}{write.xlsx}\PY{p}{(}\PY{n}{przyklad2}\PY{p}{,}\PY{n}{file}\PY{o}{=}\PY{l+s}{\PYZdq{}}\PY{l+s}{Ceny poszczególnych towarów na przestrzeni 13 lat.xlsx\PYZdq{}}\PY{p}{)}
\end{Verbatim}
\end{tcolorbox}

    \hypertarget{mamy-ceny-kazdego-towaru-w-kaux17cdym-roku-teraz-czas-na-przykux142adowy-wykres}{%
\subsubsection{Mamy ceny kazdego towaru w każdym roku, teraz czas na
przykładowy
wykres}\label{mamy-ceny-kazdego-towaru-w-kaux17cdym-roku-teraz-czas-na-przykux142adowy-wykres}}

    \hypertarget{dla-cytryny}{%
\paragraph{Dla cytryny}\label{dla-cytryny}}

    \begin{tcolorbox}[breakable, size=fbox, boxrule=1pt, pad at break*=1mm,colback=cellbackground, colframe=cellborder]
\prompt{In}{incolor}{ }{\boxspacing}
\begin{Verbatim}[commandchars=\\\{\}]
\PY{n+nf}{ggplot}\PY{p}{(}\PY{n}{przyklad2}\PY{p}{,}\PY{n+nf}{aes}\PY{p}{(}\PY{n}{x}\PY{o}{=}\PY{n}{wartosci}\PY{p}{,} \PY{n}{y}\PY{o}{=}\PY{n}{lata}\PY{p}{)}\PY{p}{)}\PY{o}{+}
  \PY{n+nf}{geom\PYZus{}line}\PY{p}{(}\PY{n+nf}{aes}\PY{p}{(}\PY{n}{x}\PY{o}{=}\PY{n+nf}{as.numeric}\PY{p}{(}\PY{n}{przyklad2}\PY{o}{\PYZdl{}}\PY{n}{czekolada.mleczna...za.100g}\PY{p}{)}\PY{p}{,} \PY{n}{colour}\PY{o}{=}\PY{l+s}{\PYZdq{}}\PY{l+s}{red\PYZdq{}}\PY{p}{)}\PY{p}{)}\PY{o}{+}
  \PY{n+nf}{ggtitle}\PY{p}{(}\PY{l+s}{\PYZdq{}}\PY{l+s}{Zmiana ceny cytryny w latach 2006\PYZhy{}2019\PYZdq{}}\PY{p}{)}\PY{o}{+}
  \PY{n+nf}{xlab}\PY{p}{(}\PY{l+s}{\PYZdq{}}\PY{l+s}{Cena w zł\PYZdq{}}\PY{p}{)}
\end{Verbatim}
\end{tcolorbox}


    % Add a bibliography block to the postdoc
    
    
    
\end{document}
